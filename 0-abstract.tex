\chapter*{Abstract}

Interactive Theorem Proving (ITP) enables researchers to build reusable libraries of formalized mathematics, algorithms, and data structures. This thesis contributes to the formalization of AVL trees in the HOL Theorem Prover (HOL4), an interactive theorem-proving system based on higher-order logic. The formalization covers the theoretical foundations of AVL trees, focusing on essential properties such as height constraints, node counts, and balancing operations through rotations after insertions and deletions. A recursive relationship between the number of nodes in an AVL tree and the Fibonacci sequence is established, demonstrating that minimal AVL trees achieve optimal performance with a height logarithmic to the number of nodes. Additionally, design adaptations made to integrate the formalization into HOL4’s framework are discussed.

Comparisons with other existing theories, such as Isabelle's AVL tree implementation and balanced binary search trees (balanced\_bst) in HOL4 are also presented. The present work aims to become a part of HOL4's official examples, adding to its valuable collection of verified data structures for future developers.

\newpage